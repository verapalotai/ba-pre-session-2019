\documentclass[10pt]{article}
 
\usepackage[margin=1in]{geometry} 
\usepackage{amsmath,amsthm,amssymb, graphicx, multicol, array, enumerate, gensymb}
\newcommand{\N}{\mathbb{N}}
\newcommand{\Z}{\mathbb{Z}}
 
\newenvironment{problem}[2][Problem]{\begin{trivlist}
\item[\hskip \labelsep {\bfseries #1}\hskip \labelsep {\bfseries #2.}]}{\end{trivlist}}

\begin{document}
 
\title{Mathematics Exam Solutions}
\date{}
\maketitle

 \section{Elementary algebra}
 
\begin{problem}{1.1}
Simplify
$$\frac{x^{32}}{x^9 \cdot x^2} \cdot \frac{x^7}{x^2} = \frac{x^{39}}{x^{13}} = x^{26}$$
\end{problem}

\begin{problem}{1.2}
Solve for $x$:
$$8^2 \cdot 4^x \cdot 2^x = 8^4$$ 
$$2^6 \cdot 2^{2x} \cdot 2^x = 2^{12} $$ 
$$2^{3x} = 2^6$$ 
$$ 3 \cdot x = 6 $$
$$ x = 2 $$
\end{problem}

\begin{problem}{1.3}
Calculate the missing value. If $\frac{x}{y}$ is 3, then $x^{-4}y^{4}=\dots$
$$ x = 3 \cdot y $$
$$ (3\cdot y)^{-4}y^4 = 3^{-4}\cdot y^{-4} \cdot y^{4} = \frac{1}{3^4} = \frac{1}{81} \approx 0.012 $$
\end{problem}

\begin{problem}{1.4}
Calculate
$$\frac{\sqrt{4^{15}}}{\sqrt{16^7}} = \sqrt{\frac{4^{15}}{4^{14}}} = \sqrt{4} = 2 $$
\end{problem}

\begin{problem}{1.5}
True or False ($x$ and $y$ and $z$ are real numbers):
\begin{enumerate}[(a)]
    \item $x+(y+z)=(y+x)+z$ $\ \ \ $   TRUE
    \item $y(x+z)=xy+zy$  $\ \ \ $  TRUE
    \item $x^{y+z}=x^z+x^y$  $\ \ \ $  FALSE
    \item $\frac{x^z}{x^y}=x^{y-z}$  $\ \ \ $  FALSE
\end{enumerate}
\end{problem}

\begin{problem}{1.6}
Find the solution set for the inequality below:
$$\ln(x) \ge e$$
$$ e^{\ln(x)} \ge e^e$$
$$ x \ge e^e $$
\end{problem}

\section{Functions of one variable}

\begin{problem}{2.1 (Based on SYD 2.5.6)}
The relationship between temperatures measured in Celsius and Fahrenheit is linear. 0\degree C is equivalent to 32\degree F and 100\degree C is the same as 212\degree F.
 Which temperature is measured by the same number on both scales? 
\medskip \\
Equations:
\begin{enumerate}[(a)]
	\item $100 = 212 \cdot a + b$
	\item $0 = 32 \cdot a + b$
\end{enumerate}
Thus $32 \cdot a = -b$ and:
$$ 100 = 212\cdot a - 32\cdot a = 180\cdot a$$
Thus $b = -32\cdot \frac{100}{180}$ and:
$$ y = \frac{100}{180} \cdot x - 32\cdot \frac{100}{180}$$
If $y = x$:
$$ x - \frac{100}{180}\cdot x = -32\cdot \frac{100}{180}$$
$$ \frac{180}{100}\cdot x - x = -32 $$
$$ x = -40$$
 
\end{problem}

\begin{problem}{2.2}
Take the following function $f(x)=3x-12$. Find y if $f(y)=0$.
$$f(y) = 3y-12 = 0$$
$$3y = 12$$
$$y = 4$$
\end{problem}

\begin{problem}{2.3}
Find all values of x that satisfy:
$$9^{x^2-6x+2}=81$$
$$9^{x^2-6x+2} = 9^2$$
$$x^2-6x+2 = 2$$
$$x^2-6x = 0$$
$$x\cdot (x-6) = 0$$
So $x_1 = 0$, $x_2 = 6$.
\end{problem}


\begin{problem}{2.5}
Calculate the following value
$$\log_{\pi}\left(\frac{1}{\pi^5} \right)$$
$$\log_{\pi}\left(\pi^{-5} \right) = -5 $$
\end{problem}

\section{Calculus}

\begin{problem}{3.1}
Calculate the following sum
$$\sum\limits_{i=0}^{\infty} \left( \frac{1}{5^i}+0.3^i\right) = \sum\limits_{i=0}^{\infty} \left( \frac{1}{5^i}\right) + \sum\limits_{i=0}^{\infty} 0.3^i$$
Applying the formula for infinite geometric series:
$$\sum\limits_{i=0}^{\infty} \left( \frac{1}{5^i}\right) = \frac{5}{4}$$
$$\sum\limits_{i=0}^{\infty} \left( \frac{3}{10} \right)^i = \frac{10}{7}$$
$$\sum\limits_{i=0}^{\infty} \left( \frac{1}{5^i}+0.3^i\right) = \frac{5}{4}+\frac{10}{7} \approx 2.67857$$
\end{problem}

\begin{problem}{3.2}
Find the following limit
$$\lim\limits_{x \rightarrow 5}\frac{x^2-25}{x-5}$$
$$\lim\limits_{x \rightarrow 5}\frac{(x-5)(x+5)}{x-5}$$
$$\lim\limits_{x \rightarrow 5}(x+5) = 10$$

\end{problem}

\begin{problem}{3.3}
Find the slope of the function $f(x)=x^3-4$ at $(-2,-12)$.
$$f'(x) = 3x^2$$
$$f'(-2) = 12 = m$$
$$y = m\cdot x$$
$$y-y_1 = m\cdot (x-x_1)$$
$$y+12 = 12\cdot(x+2)$$
$$y = 12x + 12$$
\end{problem}

\begin{problem}{3.4}
Find the derivative of the following function:
$$f(x)= \frac{x^5+3}{x^2-1}$$
$$ f'(x) = \frac{5x^4\cdot(x^2-1) - (x^5+3)\cdot 2x}{(x^2-1)^2} $$
\end{problem}

\begin{problem}{3.5}
Find the second derivative of the following function:
 $$f(x) = x^9+3$$
 $$f'(x) = 9x^8$$
 $$f''(x) = 72x^7$$
\end{problem}

\begin{problem}{3.6}
Is the function  $f(x)=\frac{1}{x}$ continuous at $0$? Why? \\
It is not because division by zero is undefined therefore as $x$ approaches 0 from the right, $y$ approaches infinity and as $x$ approaches 0 from the left, $y$ approaches negative infinity.
\end{problem}

\begin{problem}{3.7}
Consider the following function. Find all of its local minima, local maxima or inflection points. 
$$f(x)=4x^3-12x$$
$$f'(x) = 12x^2-12 = 0$$
$$ x^2 = 1 $$
$$ x = \pm 1 $$
Possible local minima/maxima at $x = \pm 1$.
$$f''(x) = 24x$$
For $x = 1$, $f''(x) = 24$ so it's a local minimum. \\
For $x = -1$, $f''(x) = -24$ so it's a local maximum.\\
The function also has an inflection point at $x = 0$.
\end{problem}

\begin{problem}{3.8}
Let $f(x,y)=x^3-y^2$. Calculate $f(2,3)$
$$ f(2,3) = 8-9 = -1 $$
\end{problem}

\begin{problem}{3.9}
Consider the following function: $f(x,y)=\ln(x-3y)$. For what combinations of $x$ and $y$ is this function defined?
\medskip \\
$$ x-3y > 0$$
$$ x>3y$$ 
\end{problem}

\begin{problem}{3.10}
Find the following partial derivative:
$$\frac{\partial}{\partial \, x} \left(x^5y^7+\frac{x^2}{y^3}\right) = \left( 5x^4y^7 + 2x\cdot \frac{1}{y^3} \right)$$

\end{problem}


\begin{problem}{3.12}
Solve the following constrained optimization problem using Lagrange's method:
$\max x^2y^2$ s.t. $2x+y=9$
\medskip \\
$$L(x,y,\lambda) = f(x,y)-\lambda g(x,y) = x^2y^2-\lambda (2x+y-9) $$
$$ \frac{\partial L}{\partial x} = 2xy^2 - 2\lambda = 0$$
$$ \frac{\partial L}{\partial y} = 2yx^2 - \lambda = 0$$
$$ \frac{\partial L}{\partial \lambda} = 2x + y -9 = 0 $$
$$ xy^2 = \lambda $$
$$ 2yx^2 = \lambda $$
$$ y=2x$$
$$ x = \frac{9}{4} $$
$$ y = \frac{9}{2} $$
\end{problem}

\section{Linear algebra}

\begin{problem}{4.1}
Take the following matrices:
$$A=\begin{bmatrix} 2 & 5\\ 2 & 1 \\ 7 & 6\end{bmatrix}$$
$$B=\begin{bmatrix} 1 & 0 & 1\\9 & 1 & 5\end{bmatrix}$$
What is $B \cdot A$?
$$B \cdot A = \begin{bmatrix} 9 & 11\\ 55 & 76 \end{bmatrix} $$
\end{problem}

\begin{problem}{4.2}
Take the following matrices:
$$A=\begin{bmatrix} 5 & 3\\ 0 & 1 \\ 1 & 2\end{bmatrix}$$
$$B=\begin{bmatrix} 8 & 4 & 0\\2 & 1 & 2\end{bmatrix}$$
What is $A \cdot B$?
$$A \cdot B = \begin{bmatrix} 46 & 23 & 6\\ 2 & 1 & 2\\12 & 6 & 4\end{bmatrix} $$
\end{problem}

\begin{problem}{4.3}
What is the transpose of the following matrix?
$$A=\begin{bmatrix}e & 93 & 4.7\\ 2 & 6.1 & 4.22 \\ 4 & \pi & 0\end{bmatrix}$$
$$A^T = \begin{bmatrix}e & 2 & 4\\ 93 & 6.1 & \pi \\ 4.7 & 4.22 & 0\end{bmatrix} $$
\end{problem}

\begin{problem}{4.4}
Calculate the determinant of
$$A=\begin{bmatrix}2 & 6 \\ 2 & 8 \end{bmatrix} $$
$$det(A) = 16-12 = 4$$
\end{problem}

\section{Probability theory}

\begin{problem}{5.1}
You run an experiment where you toss a dice two times. Each time you get either 1, 2, 3, 4, 5 or 6. What is the sample space of your experiment?
\medskip \\
Sample space: $6\cdot 6 = 36$.
\end{problem}

\begin{problem}{5.2}
Assume that in a certain country 0.1\% of the population uses a certain drug. You have a way to test drug use, which will give you a positive result in 98\% of the cases where the individual is indeed a drug user and a negative result in 99.7\% of the cases where the individual doesn't use the drug. What is the probability that someone with a positive drug test is indeed a drug user?
\medskip \\
Let's take the following events:
\begin{enumerate}[$\bullet$]
\item $A$: positive drug test
\item $A^C$: negative drug test
\item $B$: drug user, P($B$) = 0.001
\item $B^C$: not a drug user, P($B^C$) = 0.999
\end{enumerate}
$\textbf{P}(B|A) = \frac{P(A|B)\cdot P(B)}{P(A|B)\cdot P(B)+P(A|B^C)\cdot P(B^C)} = \frac{0.98\cdot 0.001}{0.98\cdot 0.001 + 0.003\cdot 0.999} = 0.2464 \rightarrow 24.64\%$
\end{problem}

\begin{problem}{5.3}
You run an experiment in which you toss a dice 20 times and record how many times you ended up with a 1, 2, 3, 4, 5 or 6. Your random variable is the number of times you ended up with a 5. What is  expected value of this random variable?
\end{problem}
$X$ = number of times we ended up with a 5 
$$E(X) = 20\cdot \frac{1}{6} \approx 3.33$$
\end{document}

